\documentclass{article}
\usepackage{hyperref}
\usepackage{natbib}
\usepackage[right=1in,top=1in,left=1in,bottom=1in]{geometry}

\begin{document}
\title{{\large Review} \\ Grip forces and load forces in handovers: Implications for designing human-robot handover controllers}
\author{Luke Fraser}
\date{\today}
\maketitle

% REFERENCE THE PAPER HERE ////////////////////////////////////////////////////////////////////
\begingroup
\renewcommand{\section}[2]{}
\bibliographystyle{plain}
\bibliography{references}
\endgroup

% /////////////////////////////////////////////////////////////////////////////////////////////
\section*{Summary}
% WRITE SUMMARY SECTION HERE //////////////////////////////////////////////////////////////////
In this paper the authors discuss the important of the etiquette of handing and receiving objects between and robot and a person. They discuss through data analysis that a general etiquette is witnessed when two people exchange objects. The \"giver\" performs the role of safety and the \"receiver\" performs the role of efficiency. This extends to say that a robot involved with human exchange of objects should follow common etiquette of exchange. The authors present results from several studies of human to human transfers. The results from the studies indicate the differences between the two roles of an exchange. They provide an in depth understanding of the exchanges and provide very useful information in order to implement such an exchange for a human to robot transfer.
% /////////////////////////////////////////////////////////////////////////////////////////////
\section*{Strengths}
% DISCUSS THE STRENGTHS OF THE PAPER //////////////////////////////////////////////////////////
The information presented will undoubtedly be useful for robots that need to exchange objects with other humans. They also presented the results in a very thorough and clear manor.
% /////////////////////////////////////////////////////////////////////////////////////////////
\section*{Critique}
% DISCUSS THE CRITIQUE OF THE PAPER ///////////////////////////////////////////////////////////
I don't think this paper required eight pages to gain the full understanding of the results from the studies. The paper goes so in depth in the description of the results in what seems like an attempt to get eight full pages out of the experiments conducted. The results are quite clear as is seen by the very short conclusion. I understood the need for the research, but did not expect such a lengthy response to simple results.
% /////////////////////////////////////////////////////////////////////////////////////////////
\cite{6249621}

\end{document}
