\documentclass{article}
\usepackage{hyperref}
\usepackage{natbib}
\usepackage[right=1in,top=1in,left=1in,bottom=1in]{geometry}

\begin{document}
\title{{\large Review} \\ Who is IT? Inferring role and intent from agent motion}
\author{Luke Fraser}
\date{\today}
\maketitle

% REFERENCE THE PAPER HERE ////////////////////////////////////////////////////////////////////
\begingroup
\renewcommand{\section}[2]{}
\bibliographystyle{plain}
\bibliography{references}
\endgroup

% /////////////////////////////////////////////////////////////////////////////////////////////
\section*{Summary}
% WRITE SUMMARY SECTION HERE //////////////////////////////////////////////////////////////////
In this paper the authors present a method for a robot to interpret and assign meaning to different scenarios witnessed of a group of people. Specifically the robot is able to interpret a group of people playing tag and assign the different roles and intentions of the players. The robot is then able to join in on the game with the people. Prior work was done by Heider and Simmel. They discovered that when people were given simple visual cues they were able to form complex stories of what was actually occurring. The robot understands the scene from a top down view of the data provided from the position of the people in the room. The data is then analyzed over time to surmise what is going on. The robot assigns sudo forces to the each persons position data witnessed over time. As the people move around the robot is able to adjust these forces and use them to interpret the scene. The computation of the forces is O(n\^2) and does not scale to very large groups. Depending on the analysis of the forces and how they change will flag different events for the robot. Based on the changes to the forces the robot is able to understand and react to different events that occur. Experiments were done to test the method, the robot was also provided with several ways to respond to different situations in the experiment. The results of the study showed that the robots understanding of a scene was the same as a humans 70.8\% of the time. While humans agree with each other only 78\% of the time.
% /////////////////////////////////////////////////////////////////////////////////////////////
\section*{Strengths}
% DISCUSS THE STRENGTHS OF THE PAPER //////////////////////////////////////////////////////////
The model used to understand events in a scene is very powerful and expandable. With prior understanding of different a robot would be able to understand and arbitrary scene with several people. This generalness of the method used is strong for handling many situations. The work of this paper could also be expanded to handling context sensitive situations that represent very complex events.
% /////////////////////////////////////////////////////////////////////////////////////////////
\section*{Critique}
% DISCUSS THE CRITIQUE OF THE PAPER ///////////////////////////////////////////////////////////
Although it is quite significant for a robot to be able to interpret cartoons the same as people it would be more useful to real world examples of the robot was shown a real world situation and the other person to compare against was given the actual images rather than cartoon images. Although the robot is views the world much simpler it would also be interesting to see how the robot compares to a person when the full images information is given to a person.
% /////////////////////////////////////////////////////////////////////////////////////////////
\cite{4354065}

\end{document}
