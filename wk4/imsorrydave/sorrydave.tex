\documentclass{article}
\usepackage{hyperref}
\usepackage{natbib}
\usepackage[right=1in,top=1in,left=1in,bottom=1in]{geometry}

\begin{document}
\title{{\large Review} \\ I'M Sorry, Dave: I'M Afraid I Won'T Do That: Social Aspects of Human-agent Conflict}
\author{Luke Fraser}
\date{\today}
\maketitle

% REFERENCE THE PAPER HERE ////////////////////////////////////////////////////////////////////
\begingroup
\renewcommand{\section}[2]{}
\bibliographystyle{plain}
\bibliography{references}
\endgroup

% /////////////////////////////////////////////////////////////////////////////////////////////
\section*{Summary}
% WRITE SUMMARY SECTION HERE /////////////////////////////////////////////////////////////////
In this paper the authors discuss and analyses the implications of a robot that disagrees with the user. Experiments were conducted that involved robots challenging user choices and suggesting alternatives. After the experiments they evaluated each user to assess the perception of the robot. The experiment had several variables, whether the robot agrees, whether the voice comes from the robot or a secondary object. The results showed that people responded differently depending on whether or not the robot disagreed with the user. Users found that they felt more similar to the robot when it agreed on every question in the study. The people also changed their response more often when the robot disagreed with the person. The results from the locations of the voice did not give relevant data.
% /////////////////////////////////////////////////////////////////////////////////////////////
\section*{Strengths}
% DISCUSS THE STRENGTHS OF THE PAPER //////////////////////////////////////////////////////////
The main contribution of this paper is the understanding the difference in perception from a robot that agrees with it's users versus a robot that challenges its users and how this effects the users choices. With this knowledge more intelligent systems can be designed that are able to provide better assistance to users by providing effective disagreement.
% /////////////////////////////////////////////////////////////////////////////////////////////
\section*{Critique}
% DISCUSS THE CRITIQUE OF THE PAPER ////////////////////////////////////////////////////////////
The paper does not evaluate all the different ways in which disagreement occurs. It would have been interesting to see the difference in response depending on the way in which the robot disagreed and whether or not they supply a counter example for the disagreement. This results from studying this could show that there are ways a reobot could disagree that may give a more positive response.
% /////////////////////////////////////////////////////////////////////////////////////////////
\cite{Takayama:2009:ISD:1518701.1519021}

\end{document}
