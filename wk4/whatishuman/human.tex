\documentclass{article}
\usepackage{hyperref}
\usepackage{natbib}
\usepackage[right=1in,top=1in,left=1in,bottom=1in]{geometry}

\begin{document}
\title{{\large Review} \\ What is a Human? - Toward Psychological Benchmarks in the Field of Human-Robot Interaction}
\author{Luke Fraser}
\date{\today}
\maketitle

% REFERENCE THE PAPER HERE ////////////////////////////////////////////////////////////////////
\begingroup
\renewcommand{\section}[2]{}
\bibliographystyle{plain}
\bibliography{references}
\endgroup

% /////////////////////////////////////////////////////////////////////////////////////////////
\section*{Summary}
% WRITE SUMMARY SECTION HERE //////////////////////////////////////////////////////////////////
In this paper the authors try to define a human benchmark for different studies. There are several different scenarios that the authors try to address. The different benchmarks include when the idea is that the robot is considered human, or the robot is interacting with humans. These benchmarks are important so that many different studies can compare their data against these benchmarks. Without a benchmark there is no way in which different experiments can be compared or analyzed against each other. The paper then discusses the different types of benchmarks that should be created in the field of HRI. They discuss how the amount of benchmarks is relatable to how well the field has captured human behavior ans response.
% /////////////////////////////////////////////////////////////////////////////////////////////
\section*{Strengths}
% DISCUSS THE STRENGTHS OF THE PAPER //////////////////////////////////////////////////////////
The paper was able to adequately define the different areas of a benchmark that would qualify the results. Given the qualities suggested better benchmarks could be developed that will provide a useful comparison against other studies. This will allow researchers to generate better benchmarks in the field of HRI.
% /////////////////////////////////////////////////////////////////////////////////////////////
\section*{Critique}
% DISCUSS THE CRITIQUE OF THE PAPER ///////////////////////////////////////////////////////////
The main problem with the benchmark description is that there is no backing for the opinions expressed on proper benchmarking. Without validation of the advice for benchmarks there is no way of showing that a good benchmark can be implemented based on the paper.
% /////////////////////////////////////////////////////////////////////////////////////////////
\cite{4107835}

\end{document}
