\documentclass{article}
\usepackage{hyperref}
\usepackage{natbib}
\usepackage[right=1in,top=1in,left=1in,bottom=1in]{geometry}

\begin{document}
\title{{\large Review} \\ Governing Lethal Behavior: Embedding Ethics in a Hybrid Deliberative/Reactive Robot Architecture}
\author{Luke Fraser}
\date{\today}
\maketitle

% REFERENCE THE PAPER HERE ////////////////////////////////////////////////////////////////////
\begingroup
\renewcommand{\section}[2]{}
\bibliographystyle{plain}
\bibliography{references}
\endgroup

\section*{Summary}
% WRITE SUMMARY SECTION HERE //////////////////////////////////////////////////////////////////
In this paper the authors discuss the ethics of robots and lethal behavior and whether or not a robot that performs lethal action falls into the laws of war. The paper was funded by the US Army to ensure that robots behave in a legal manner with respect to military action. The goal of this paper was to design an \emph{artificial conscience} for autonomous robotic systems. the paper discusses the laws of war and the rules of engagement as they have taken apart of history as well as their importance. The current state of the art reflecting lethal robots has typically been justified by human involvement. Meaning that when a robot is going to take lethal action a human person is always there to oversee the and confirm the decisions made by the robot. The paper then lists several systems that are being made that possibly violate the laws of war in different ways. The authors propose ethical issues to consider when developing lethal systems. The paper goes on to discuss philosophical thought on the issues of the lethal autonomous robot. the paper then outlines future work of developing ethical systems that ethically outperform their human counterparts.
% /////////////////////////////////////////////////////////////////////////////////////////////
\section*{Strengths}
% DISCUSS THE STRENGTHS OF THE PAPER //////////////////////////////////////////////////////////
The paper develops their baseline for ethical war decisions from the Law of War and Rules of engagement and they attempted to build ethical war decisions based on the most widely accepted and unbiased source.
% /////////////////////////////////////////////////////////////////////////////////////////////
\section*{Critique}
% DISCUSS THE CRITIQUE OF THE PAPER ///////////////////////////////////////////////////////////
I don't see how any attempted of programming in ethics is valid as an overall scheme to create ethical robots. Ethics is a very dynamic and location dependent feature of humanity. If we are to expect robots to be ethical then ethical decisions should at least at the start be based on the country they are developed in. In the future though I think that robotic systems in order to be ethical will require intelligence similar to humans. The idea of a single system controlling ethics in my mind proposes ethical issues by itself. Also the idea in the future to produce more ethical robots than humans seems to me to be a fallacy. two pairs of robots if they are truly ethical robots should no have perfectly matching ethical decisions. The feel is that the robot will simply be overall ethical with respect to external view alone.
% /////////////////////////////////////////////////////////////////////////////////////////////
\cite{Arkin07governinglethal}

\end{document}
