\documentclass{article}
\usepackage{hyperref}
\usepackage{natbib}
\usepackage[right=1in,top=1in,left=1in,bottom=1in]{geometry}

\begin{document}
\title{{\large Review} \\ Relational artifacts with children and elders: The complexities of cybercompanionship}
\author{Luke Fraser}
\date{\today}
\maketitle

% REFERENCE THE PAPER HERE ////////////////////////////////////////////////////////////////////
\begingroup
\renewcommand{\section}[2]{}
\bibliographystyle{plain}
\bibliography{references}
\endgroup

% /////////////////////////////////////////////////////////////////////////////////////////////
\section*{Summary}
% WRITE SUMMARY SECTION HERE //////////////////////////////////////////////////////////////////
In this paper the authors discuss the psychological effects of the robots as cybercompanions and peoples views of robots as ``\emph{alive}''. The paper uses Furby's, and My Real Babies as examples of commercially available robots that are companions to children. These robots interacted with children and many kids developed relationship with them. The paper then evaluates the modern trends in relational artifacts and that people are more interested in having relational artifacts that are pets. In the next section the paper describes how children respond to companion robots. Orelia a child subject of theirs responds to Aibo believing that it is impossible for Aibo to love because it's love is artificial. Orelia did not think that it was worth her time to invest in something that could not love you back. This paper then continues to discuss many other scenarios dealing with people and their relationships with robots.
% /////////////////////////////////////////////////////////////////////////////////////////////
\section*{Strengths}
% DISCUSS THE STRENGTHS OF THE PAPER //////////////////////////////////////////////////////////
This paper evaluates many different scenarios and the effects and relationships of the relational artifacts. The paper shows a modern understanding of how relational artifacts are viewed by people.
% /////////////////////////////////////////////////////////////////////////////////////////////
\section*{Critique}
% DISCUSS THE CRITIQUE OF THE PAPER ///////////////////////////////////////////////////////////
I don't think that the conclusions that the authors arrived at can be done with such a small sample of random observations with people and their robots. The paper itself was difficult to read and it was written in a way that made it hard to understand the conclusions they were making. The paper does not offer much of a contribution to the state of the art. The paper does develop a system that would help create a response of ``life'' from robots.
% /////////////////////////////////////////////////////////////////////////////////////////////
% \cite{goossens93}

\end{document}
