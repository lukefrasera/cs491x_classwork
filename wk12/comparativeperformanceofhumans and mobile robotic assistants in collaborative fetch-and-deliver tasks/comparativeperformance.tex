\documentclass{article}
\usepackage{hyperref}
\usepackage{natbib}
\usepackage[right=1in,top=1in,left=1in,bottom=1in]{geometry}

\begin{document}
\title{{\large Review} \\ Comparative Performance of Human and Mobile Robotic Assistants in Collaborative Fetch-and-deliver Tasks}
\author{Luke Fraser}
\date{\today}
\maketitle

% REFERENCE THE PAPER HERE ////////////////////////////////////////////////////////////////////
\begingroup
\renewcommand{\section}[2]{}
\bibliographystyle{plain}
\bibliography{references}
\endgroup

% /////////////////////////////////////////////////////////////////////////////////////////////
\section*{Summary}
% WRITE SUMMARY SECTION HERE //////////////////////////////////////////////////////////////////
In this paper the authors discuss the effect of using socially aware path planning with Assistive free-flyers to help people better understand the intent of the AFFs. A modified path planning system is presented as wells as two studies that evaluate the effectiveness of the the new modified flight planner. The motivation of the paper is to provide more socially aware flying to aerial drones that work closely with people. Because small aerial drones work in close proximity to people they will need to communicate better with people and fly in a more human understandable manner. People need to understand instinctively how the drones will fly in order to promote a good report between people. The authors define several aspects of AFFs that need to be considered when developing flight paths. The use of several manipulations the be used to reflect different behavior, arcing, easing, and anticipation. A study was performed to understand the impact of using the different flight manipulators. The study was evaluated and hypotheses were created for the second study. The idea was that using the modified flight paths people would more intuitively understand robot intent. The results of the study showed that the people were more able to predict the intent of the robot through the use of flight manipulators. As such all 3 hypotheses were validated by the study. People would feel more safe and would be more willing to work with aerial robots that used the manipulated flight paths.
% /////////////////////////////////////////////////////////////////////////////////////////////
\section*{Strengths}
% DISCUSS THE STRENGTHS OF THE PAPER //////////////////////////////////////////////////////////
The use of modified flight paths is an extremely useful aspect to consider when developing robots that work closely with people. The consideration of flight intent will help people to understand what a robot is doing at any given moment as well as understand exactly how to work side by side with a robot that exhibits proper social flight cues. This paper hosts a very important aspect of aerial robotics that will greatly improve the use of aerial robot human interactions. It would be very interesting to see future work from this paper to explore more complex intent recognitions challenges, to see if there are more complex tasks that could be recognized naturally by humans. I really liked this paper!!.
% /////////////////////////////////////////////////////////////////////////////////////////////
\section*{Critique}
% DISCUSS THE CRITIQUE OF THE PAPER ///////////////////////////////////////////////////////////
The second study although it validates the experiment quite well it would be interesting to see the use of the flight manipulators in a real world environment where the robot is assisting people and properly conveying intent while flying. This would further show the benefits of the manipulators and there effects on human comfort.
% /////////////////////////////////////////////////////////////////////////////////////////////
\cite{Unhelkar:2014:CPH:2559636.2559655}

\end{document}
