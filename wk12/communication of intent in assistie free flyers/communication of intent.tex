\documentclass{article}
\usepackage{hyperref}
\usepackage{natbib}
\usepackage[right=1in,top=1in,left=1in,bottom=1in]{geometry}

\begin{document}
\title{{\large Review} \\ Communication of Intent in Assistive Free Flyers}
\author{Luke Fraser}
\date{\today}
\maketitle

% REFERENCE THE PAPER HERE ////////////////////////////////////////////////////////////////////
\begingroup
\renewcommand{\section}[2]{}
\bibliographystyle{plain}
\bibliography{references}
\endgroup

% /////////////////////////////////////////////////////////////////////////////////////////////
\section*{Summary}
% WRITE SUMMARY SECTION HERE //////////////////////////////////////////////////////////////////
In this paper the authors discuss the use of assistive robots in industrial environments. The idea is instead of solely using robots to automate the building process there is a push to use assistive robots to help the process become more efficient and less monotonous. The authors discuss the use of n assistive robot that brings equipment to people to alleviate the need for every person to pick up their parts everyday. The goal of this paper is to prevent people from performing non-valued tasks. In this paper they analyze the effectiveness and efficiency of a robots assisting in a fetch-and-deliver environment. A fetch-and-deliver environment is as it sounds an environment where a robot brings necessary items to people such that the person does not have to. A likert scale was used to evaluate the effects of using a robot in the study.
% /////////////////////////////////////////////////////////////////////////////////////////////
\section*{Strengths}
% DISCUSS THE STRENGTHS OF THE PAPER //////////////////////////////////////////////////////////
The experiment design was developed in a thorough way to understand the differences between human and robot assistants. There use of factory noise was an interesting insight into preventing position prediction of the robot based on the sound that it makes.
% /////////////////////////////////////////////////////////////////////////////////////////////
\section*{Critique}
% DISCUSS THE CRITIQUE OF THE PAPER ///////////////////////////////////////////////////////////
The current state of the art prevents good hand offs from being executed by robots along with other things. The particular robot chosen for this study does not attempt to bridge this gap either. This robot merely provides a moving platform to deliver items to people. This robot does not provide a general baseline for the effectiveness of robots working in industrial environments. It only provides information on a very small gamut of necessary assistive tasks in factories. This robot would not eliminate the need for humans to bring other people items in factory environments.
% /////////////////////////////////////////////////////////////////////////////////////////////
\cite{Szafir:2014:CIA:2559636.2559672}

\end{document}
