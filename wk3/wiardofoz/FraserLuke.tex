\documentclass{article}
\usepackage{hyperref}
\usepackage{natbib}
\usepackage[right=1in,top=1in,left=1in,bottom=1in]{geometry}

\begin{document}
\title{{\large Review} \\ The Oz of Wizard: Simulating the Human for Interaction Research}
\author{Luke Fraser}
\date{\today}
\maketitle

% REFERENCE THE PAPER HERE ////////////////////////////////////////////////////////////////////
\begingroup
\renewcommand{\section}[2]{}
\bibliographystyle{plain}
\bibliography{references}
\endgroup

% /////////////////////////////////////////////////////////////////////////////////////////////
\section*{Summary}
% WRITE  SUMMARY SECTION HERE /////////////////////////////////////////////////////////////////
In this paper the authors evaluate the Oz of Wizard ( the reverse of the wizard of OZ  routine) method and defend the important of the its results. They propose that the results from a OZ of Wizard experiment are acceptable and have merit in understanding human response to experiments. They further breakdown the problem into many subcategories, Oz of Wizard, Oz with Wizard, Wizard with Oz, Wizard of Oz, Wizard nor Oz, and Wizard and Oz. Each of the mentioned studies evaluates differently and has different requirement on humans models as well as robot models. The use of the different models will help researches to avoid limitations of human trials and studies. This will allow robotics to progress faster and with less expenses. The paper continues to explain the different types of experiments and where and when they can and should be used to provide relevant data.
% /////////////////////////////////////////////////////////////////////////////////////////////
\section*{Strengths}
% DISCUSS THE STRENGTHS OF THE PAPER //////////////////////////////////////////////////////////
The main contribution of the paper is the added classification of the experimental studies performed in robotics. Given the larger spectrum of experimental analysis a researcher has the ability, depending on his situation, to perform different studies to obtain results without the need of long extended human trials. This is a very important contribution to HRI because evaluating a theory seems to be the hardest and most expensive factor in research.
% /////////////////////////////////////////////////////////////////////////////////////////////
\section*{Critique}
% DISCUSS THE CRITIQUE OF THE PAPER ////////////////////////////////////////////////////////////
The paper does not address or attempt to recognize problems with their proposed spectrum of experimental analysis. It would be nice to see a personal critique of their idea as well show possible situations when using the different studies will provide bad data. This information would be equally as valuable as all the positive information explained in the paper.	
% /////////////////////////////////////////////////////////////////////////////////////////////
\cite{Steinfeld_2009_6322}

\end{document}
