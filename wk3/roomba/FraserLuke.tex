\documentclass{article}
\usepackage{hyperref}
\usepackage{natbib}
\usepackage[right=1in,top=1in,left=1in,bottom=1in]{geometry}

\begin{document}
\title{{\large Review} \\ Service Robots in the Domestic Environment: A Study of the roomba Vacuum in the Home}
\author{Luke Fraser}
\date{\today}
\maketitle

% REFERENCE THE PAPER HERE ////////////////////////////////////////////////////////////////////
\begingroup
\renewcommand{\section}[2]{}
\bibliographystyle{plain}
\bibliography{references}
\endgroup

% /////////////////////////////////////////////////////////////////////////////////////////////
\section*{Summary}
% WRITE  SUMMARY SECTION HERE /////////////////////////////////////////////////////////////////
In this paper the authors evaluate the effects of consumer robots on domestic house holds. Specifically the studies that were performed involved the Roomba robot. The Roomba is a vacuum cleaner robot that autonomously cleans the floors of a house and is able to recharge itself after doing so. The study evaluated many aspects of the effects that a robot has on a household. The studies were broken down into a set of interviews that were given over different time intervals during the study. One study analyzed the effects of people expectations of domestic robots. People's expectation of robotic intelligence always emphasized learning, and because of this the Roomba never lived up to the expectations of the users. Even so this did not effect people's use of the product. They were in many ways still surprised at its performance. The paper goes on to discuss many other studies that explore other effects of the Roomba robot on a household environment. The paper ends with a discussion on the findings from the studies.
% /////////////////////////////////////////////////////////////////////////////////////////////
\section*{Strengths}
% DISCUSS THE STRENGTHS OF THE PAPER //////////////////////////////////////////////////////////
The main contribution of this paper is in the detailed analysis of the effects and expectation of a household robot. The studies and questions provide useful information that will help people design robots that can perform better in their environment and meet the expectations of the users. This paper helps understand the social environment that a domestic robot will need to fit into and shows necessary requirements of future household consumer robots.
% /////////////////////////////////////////////////////////////////////////////////////////////
\section*{Critique}
% DISCUSS THE CRITIQUE OF THE PAPER ////////////////////////////////////////////////////////////
The studies performed were limited the responses to the Roomba robot. This robot represents such a small part of HRI that I don't see how it has enough merit to acquire all of this information. The reactions to a \"smarter robot\", I believe, would provide far different results. Understanding the reactions of people when the robot in question maybe over qualified and how this effects their emotional response would also be useful information. Overall the paper needed more information from other robots to make broad claims. However this is not to say that their studies were not important or didn't provide useful information.
% /////////////////////////////////////////////////////////////////////////////////////////////
\cite{Forlizzi:2006:SRD:1121241.1121286}

\end{document}
