\documentclass{article}
\usepackage{hyperref}
\usepackage{natbib}
\usepackage[right=1in,top=1in,left=1in,bottom=1in]{geometry}

\begin{document}
\title{{\large Review} \\ The crying shame of robot nannies: an ethical appraisal}
\author{Luke Fraser}
\date{\today}
\maketitle

% REFERENCE THE PAPER HERE ////////////////////////////////////////////////////////////////////
\begingroup
\renewcommand{\section}[2]{}
\bibliographystyle{plain}
\bibliography{references}
\endgroup

% /////////////////////////////////////////////////////////////////////////////////////////////
\section*{Summary}
% WRITE SUMMARY SECTION HERE //////////////////////////////////////////////////////////////////
In this paper the authors discuss the ethical issues brought up by use of robotic nannies and how it effects children being raised by robots. The authors introduce the subject through different examples of robotic nannies and the experiences children have with them. The first issue brought up is privacy. It has become more prominent that children are under constant surveillance. The implications of the constant supervision will effect the child as he grows older socially. The idea of whether a robot should intervene in situations with the child is another ethical question brought up by the paper. It is unclear the ethical boundary that determines if a robot should be allowed to control a child or save his life from danger. The paper then goes on to discuss the social needs of robot child interaction and how to maintain longterm interest for children. The need to maintain a substantial relationship with the child becomes important in order to make successful child care robot. The next section of the paper discusses the physiological effects of child care robots on children and the potential dangers of raising children with current robot technology. Overall the paper discusses issues brought to mind when considering child care provided by robots.
% /////////////////////////////////////////////////////////////////////////////////////////////
\section*{Strengths}
% DISCUSS THE STRENGTHS OF THE PAPER //////////////////////////////////////////////////////////
This paper thoroughly evaluates the area of robot nannies and the pros and cons created by the using robots to raise children. The paper discusses in great detail all of the social aspects to consider when using a robot to interact with children. The paper also brings up the sensitive and controversial nature of using robots to raise children presenting many ethical questions to the reader.
% /////////////////////////////////////////////////////////////////////////////////////////////
\section*{Critique}
% DISCUSS THE CRITIQUE OF THE PAPER ///////////////////////////////////////////////////////////
Although the paper discusses many issues and presents ethics to consider when considering robots as caregivers for children I don't see the contribution that this paper is making. It feels that the authors simply thought about issues of robot nannies and discusses some ethical issues with it. The feel is that the paper attempted to address the entire scope, but did not do so in a flowing manner. It felt as if the paper was presented robotic nannies as a commonly used method for raising children and that parents need to protect children against ethical dangers of them.
% /////////////////////////////////////////////////////////////////////////////////////////////
\cite{Sharkey10thecrying}

\end{document}
