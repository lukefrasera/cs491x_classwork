\documentclass[10pt,a4paper]{article}
\usepackage[utf8]{inputenc}
\usepackage{amsmath}
\usepackage{amsfonts}
\usepackage{amssymb}
\usepackage{graphicx}
\usepackage[left=2cm,right=2cm,top=2cm,bottom=2cm]{geometry}
\author{Luke Fraser}
\title{Robots for Humanity: Review}
\begin{document}
\maketitle
\section{introduction}
\begin{itemize}
	\item \large{Asistive} robotics: Developing Assistive Mobile Manipulators(AMMs).
	\item \large{AMMs} functions as surrogates for the severely disabled.
	\item \large{Motivation}: To help the severely disabled to gain back some ``\emph{independence}''.
	\item \large{Problem}: The current method for providing care to the disabled is with one on one human care. A person assists the patient with day to day tasks and communicates through very limited means by using a \emph{speech board}.
	\item \large{Goal}: Create a system that is able to assist a handicapped person in day to day tasks such as shaving, scratching, cleaning, and projecting personal animated gestures to a robot. For the current client.
\end{itemize}

\section{summary}
outline:
\begin{itemize} 
	\item Assistive Mobile Manipulation
	\item Shared Autonomy
	\item User-Centered Design
	\item Assistance with Manipulation Near One's Body
	\item Assistance with Object manipulation
	\item Assistance with Social Interaction
	\item Accessible Run-Stop for AMMs
	\item Conclusion
\end{itemize}
\subsection{Assistive Mobile Manipulation}
This is the central idea of this paper. The AMM's to be developed to assist handicapped individuals and in the future the hope is that the AMM's could be and extension of the individual and function to provide complete \emph{independence}.
\subsection{Shared Autonomy}
The hypothesis was that through shared autonomy of a person working with a robot the ability to complete complicated tasks would become easier using current day technologies.
\subsection{User-Centered Design}
All of the applications that were implemented were built with the user in mind. This involved the use of questionnaires to provide ideas for the necessary applications to be used.
\subsection{Assistance with Manipulation Near One's Body}
The paper discusses the importance and issues of developing applications for assistance in near body manipulation. This sections discusses the shaving application used to help the patient shave on his own. The application discovered the need to manage the applied pressure of a razor on the users face to prevent injury.
\subsection{Assistance with Object Manipulation}
This section discusses the use of a GUI to manipulate the PR2 robot to assist with tasks around the house. As well as using semi-autonomous applications to pick up objects.
\subsection{Assistance with Social Interaction}
The user wanted to be able to perform a stand up comedy routine however this section brings up the idea that a person will want to articulate complex expressions that require specific animated gestures. They developed an animation system to build up a library of gestures to be used by the user. The system seems very slow and not a good long term system to perform animated tasks. Very similar to current keyframe animation techniques that take hour and weeks to create believable complex human like gestures. On top of this in order create human like gestures a person goes through years of training to do key frame animation.
\subsection{Accessible Run-Stop for AMMs}
This section of the paper deals with using wincing detection as a means to cause an \emph{E-Stop} to occur. The users face is tracked an when robot does something that causes a wince it will signal the robot to shut down. This alleviates the need for a physical eStop button to pressed by the user.
\subsection{conclusion}
\end{document}