\documentclass{article}
\usepackage{hyperref}
\usepackage{natbib}
\usepackage[right=1in,top=1in,left=1in,bottom=1in]{geometry}

\begin{document}
\title{{\large Review} \\ Robots for humanity: using assistive robotics to empower people with disabilities}
\author{Luke Fraser}
\date{\today}
\maketitle

% REFERENCE THE PAPER HERE ////////////////////////////////////////////////////////////////////
\begingroup
\renewcommand{\section}[2]{}
\bibliographystyle{plain}
\bibliography{references}
\endgroup

% /////////////////////////////////////////////////////////////////////////////////////////////
\section*{Summary}
% WRITE SUMMARY SECTION HERE //////////////////////////////////////////////////////////////////
In this paper the author create a robotic system designed to help severely handicapped individuals gain back some ``\emph{independence}'' and mobility through the PR2 robot. As well as providing assistance to the handicapped the robot was also designed with the caregiver in mind, to also assist caregivers in their tasks interacting with the patient. The authors questioned the caregiver and the patient to determine problems that each user would like assistance with and new activities that the patient has always wanted to do, but could not due to his or her handicap. They developed several programs to assist with specific tasks, such as shaving, communication, moving about the house, and scratching the patients face. Each system was designed based on the questionnaire requests. After each implementation new problems were discovered and each program was iteratively improved to cope with challenges encountered after tests with the patient. The paper concluded with the discussion of the successful techniques developed and the overall assistive improvement gained with the PR2 system.
% /////////////////////////////////////////////////////////////////////////////////////////////
\section*{Strengths}
% DISCUSS THE STRENGTHS OF THE PAPER //////////////////////////////////////////////////////////
The paper address a very difficult problem to help severely handicapped individuals with extreme mobility problems. The paper hopes to create a system or a start of system that will provide mobility to the handicap through a robot interface. Essentially projecting a person in the PR2 robot resulting in the patient having a new body that is more mobile than themselves. Allowing them to perform tasks otherwise impossible for them to complete with out personal assistance. Several problems that they addressed will bring a sense of independence back to the user and help transfer some of the load off of the caregiver.
% /////////////////////////////////////////////////////////////////////////////////////////////
\section*{Critique}
% DISCUSS THE CRITIQUE OF THE PAPER ///////////////////////////////////////////////////////////
The system designed in this paper is not built in a generalizable way. It was if the authors simply found problems and built several very small and specific applications that only help the person they were developed for. Based on their design process for the system to be developed for another person each application would need to be redeveloped with a new interface in mind that conforms to the new individuals handicap. Also each application was not integrated into an entire system. Several applications running on the robot do not generate a fluid system to be used by the patient, making it difficult to perform a sequence of tasks while switching between several different small scoped programs.
% /////////////////////////////////////////////////////////////////////////////////////////////
\cite{6476704}

\end{document}
