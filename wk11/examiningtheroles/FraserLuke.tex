\documentclass{article}
\usepackage{hyperref}
\usepackage{natbib}
\usepackage[right=1in,top=1in,left=1in,bottom=1in]{geometry}

\begin{document}
\title{{\large Review} \\ Dry Your Eyes: Examining the Roles of Robots for Childcare Applications}
\author{Luke Fraser}
\date{\today}
\maketitle

% REFERENCE THE PAPER HERE ////////////////////////////////////////////////////////////////////
\begingroup
\renewcommand{\section}[2]{}
\bibliographystyle{plain}
\bibliography{references}
\endgroup

% /////////////////////////////////////////////////////////////////////////////////////////////
\section*{Summary}
% WRITE SUMMARY SECTION HERE //////////////////////////////////////////////////////////////////
In this paper the authors respond to the Sharkey \& Sharkey paper on the ethics of robots in child care situations. The paper disagrees with many of the social ethics issues presented in the Sharkey paper. The paper specifically argues against the ideas that robots will remain socially unaware and psychology damage children. they presented results that robots through research will become more socially aware and that many of the concerns in the paper are misplaced. The paper goes on to discuss how in many of the scenarios discussed in the Sharkey paper many of the parents were mislead to believe that modern robots are capable of maintaining a healthy social relationship with children when clearly it wasn't the case, nor is it today. The paper defends against the viability of socially assistive robotics as a whole.
% /////////////////////////////////////////////////////////////////////////////////////////////
\section*{Strengths}
% DISCUSS THE STRENGTHS OF THE PAPER //////////////////////////////////////////////////////////
The paper adequately challenges and defends the future social capabilities of robots and their interactions with children. The discussion is also presented in a logical manner that reflects issues of the sharkey paper well.
% /////////////////////////////////////////////////////////////////////////////////////////////
\section*{Critique}
% DISCUSS THE CRITIQUE OF THE PAPER ///////////////////////////////////////////////////////////

% /////////////////////////////////////////////////////////////////////////////////////////////
\cite{Feil-seifer_dryyour}

\end{document}
