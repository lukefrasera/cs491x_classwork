\documentclass{article}
\usepackage{hyperref}
\usepackage{natbib}
\usepackage[right=1in,top=1in,left=1in,bottom=1in]{geometry}

\begin{document}
\title{{\large Review} \\ Perceptions of ASIMO: An exploration on co-operation and competition with humans and humanoid robots}
\author{Luke Fraser}
\date{\today}
\maketitle

% REFERENCE THE PAPER HERE ////////////////////////////////////////////////////////////////////
\begingroup
\renewcommand{\section}[2]{}
\bibliographystyle{plain}
\bibliography{references}
\endgroup

% /////////////////////////////////////////////////////////////////////////////////////////////
\section*{Summary}
% WRITE  SUMMARY SECTION HERE /////////////////////////////////////////////////////////////////
In this paper the author's studied the affects of cooperative vs competitive play on perceived intelligence and involvement in the robot. 26 undergraduate students (10 male, 16 female) participated in the experiment. The experiment consisted of comparing the results from a competitive game and a cooperative game. The competitive game involved the participant competing with the robot in a video-game. Whereas the cooperative game the participant worked with the robot to achieve a goal and the final score was a combined score. The results of the experiment showed that people's perception of robots differs depending on whether they were competing versus cooperating with the robots. Overall the only visible effect on perception was witnesses from males and not from females. Males were more likely to find the robot intelligent when cooperating versus competitive play where they were more involved in the task.
% /////////////////////////////////////////////////////////////////////////////////////////////
\section*{Strengths}
% DISCUSS THE STRENGTHS OF THE PAPER //////////////////////////////////////////////////////////
The contribution of this paper is how they develop interactions with humans and robots that will benefit a given task. In the case of males when involvement in a task is the most important than competitiveness with robots will produce better results. In the case of wanting the perception of intelligence cooperative play will produce better results.
% /////////////////////////////////////////////////////////////////////////////////////////////
\section*{Critique}
% DISCUSS THE CRITIQUE OF THE PAPER ////////////////////////////////////////////////////////////
There are many weaknesses that come with this paper. The sample space used is far too small to obtain very useful information as well as only coming from selection of college students around 21 years of age. The sample space is also gender biased making claims about gender difficult. Overall the selected sample was far too small.
% /////////////////////////////////////////////////////////////////////////////////////////////
\cite{Mutlu06perceptionsof}

\end{document}
