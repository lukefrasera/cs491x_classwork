\documentclass{article}
\usepackage{hyperref}
\usepackage{natbib}
\usepackage[right=1in,top=1in,left=1in,bottom=1in]{geometry}

\begin{document}
\title{{\large Review} \\ Robots in Organizations: The Role of Work flow, Social, and Environmental Factors in Human-robot Interaction}
\author{Luke Fraser}
\date{\today}
\maketitle

% REFERENCE THE PAPER HERE ////////////////////////////////////////////////////////////////////
\begingroup
\renewcommand{\section}[2]{}
\bibliographystyle{plain}
\bibliography{references}
\endgroup

% /////////////////////////////////////////////////////////////////////////////////////////////
\section*{Summary}
% WRITE SUMMARY SECTION HERE /////////////////////////////////////////////////////////////////
In this paper the authors test the implications of service robot working in a hospital environment. The service was robot was designed to deliver different items to patients rooms to assist nurses and different departments of the hospital. They studied the robot and the interactions that it had with people to understand the perception and future requirements for a hospital service bot. They studied the robot working in high intensity nursing situations as well as a low intensity nursing environment and collected data by shadowing the robot in the different environments as well as surveying the people about their experience with the robot. They perception of the robot was very different in the high intensity environment compared to the low intensity environment. In the low intensity postpartum environment the robot was received better. The nursing were more willing to work with the robot and they felt that the robot was helpful. Whereas in the high intensity environment the robot was viewed often as a burden and the nurses became very angry with the robot.
% /////////////////////////////////////////////////////////////////////////////////////////////
\section*{Strengths}
% DISCUSS THE STRENGTHS OF THE PAPER //////////////////////////////////////////////////////////
The paper contributes to the understanding of thoughts that should be considered when designing a robot for different types of environments. People react to a service robot depending on willingness of a work-flow change as well as differences in environment. In the high stress environment the staff did not have the patients to deal with the robot. Whereas in postpartum there is little stress and the robot is accepted.
% /////////////////////////////////////////////////////////////////////////////////////////////
\section*{Critique}
% DISCUSS THE CRITIQUE OF THE PAPER ////////////////////////////////////////////////////////////
The authors data collection techniques pose a problem when unbiased results are trying to be collected. If one of the experimenters is aware of a researcher near by there interactions to the robot might be different.
% /////////////////////////////////////////////////////////////////////////////////////////////
\cite{Mutlu:2008:ROR:1349822.1349860}

\end{document}
