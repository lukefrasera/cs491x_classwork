\documentclass{article}
\usepackage{hyperref}
\usepackage{natbib}
\usepackage[right=1in,top=1in,left=1in,bottom=1in]{geometry}

\begin{document}
\title{{\large Review} \\ The Snackbot: Documenting the design of a robot for long-term Human-Robot Interaction}
\author{Luke Fraser}
\date{\today}
\maketitle

% REFERENCE THE PAPER HERE ////////////////////////////////////////////////////////////////////
\begingroup
\renewcommand{\section}[2]{}
\bibliographystyle{plain}
\bibliography{references}
\endgroup

% /////////////////////////////////////////////////////////////////////////////////////////////
\section*{Summary}
% WRITE  SUMMARY SECTION HERE /////////////////////////////////////////////////////////////////
In this paper the author's describe and discuss the design process of their longterm service robot snackbot. They address design decisions that focus on the requirements for proper social interactions of real world robots. Robots working in real world environments must respond to people and interact in a socially expected way. Snackbot was developed to answer questions about how robots are perceived over long periods of time. The ultimate goal was to develop a robot that is able to have positive interactions with people while caring out its tasks. Snackbot will interact in a work environment around many employees and deliver snacks throughout the day. The rest of the paper discusses the many design decisions as well as techniques used to organize the many team members. Finally they discuss the lessons learned throughout the two year design process and what decisions are important to providing a useful long term robot.
% /////////////////////////////////////////////////////////////////////////////////////////////
\section*{Strengths}
% DISCUSS THE STRENGTHS OF THE PAPER //////////////////////////////////////////////////////////
The authors go into great detail of the design process which will benefit anyone attempting to design a long term service robot for themselves. The paper gives great insight into a refined design process for creating interactive robots. They used studies on people to determine key implementation features to provide the best experience for every person. The insights and lessons contribute to the field of robotic design and implementation on the large scale, shedding light and compartmentalizing the process of building a very complex and successful robot.
% /////////////////////////////////////////////////////////////////////////////////////////////
\section*{Critique}
% DISCUSS THE CRITIQUE OF THE PAPER ////////////////////////////////////////////////////////////
It would have been nice to see a more formal analyses of the studies they performed to determine critical design decisions. Results from the studies would show good information to the reader. The studies that were performed were not very complete and might not provide the expected results.
% /////////////////////////////////////////////////////////////////////////////////////////////
\cite{Lee_2009_6317}

\end{document}
