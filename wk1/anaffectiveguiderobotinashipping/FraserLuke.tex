\documentclass{article}
\usepackage{hyperref}
\usepackage{natbib}
\usepackage[right=1in,top=1in,left=1in,bottom=1in]{geometry}

\begin{document}
\title{{\large Review} \\ An affective guide robot in a shopping mall}
\author{Luke Fraser}
\date{\today}
\maketitle

% REFERENCE THE PAPER HERE ////////////////////////////////////////////////////////////////////
\begingroup
\renewcommand{\section}[2]{}
\bibliographystyle{plain}
\bibliography{references}
\endgroup

% /////////////////////////////////////////////////////////////////////////////////////////////
\section*{Summary}
% WRITE  SUMMARY SECTION HERE /////////////////////////////////////////////////////////////////
In this paper the authors discuss the design and studies of a shopping mall robot. The robot operated for 25 days in a shopping mall and interacted with many mall customers during that time period. The robot was intended to build report with its users by storing the id of previous users with an RFID tag. The robot would assist people with questions about the mall as well as provide useful sales suggestions. The robot used a wizard of oz technique to respond to questions. This made it possible to handle speech recognition in a noisy environment as well preventing the need to worry about fully autonomous conversations. The results of the paper showed that the users fond the robot helpful and responded well the the advertisements given by the robot. The recurring visits didn't not happen with enough of the participants to conclude anything. The robot received positive recognition from the users throughout the study.
% /////////////////////////////////////////////////////////////////////////////////////////////
\section*{Strengths}
% DISCUSS THE STRENGTHS OF THE PAPER //////////////////////////////////////////////////////////
The main contribution is the discussion on the implementation of a robot designed for commercial use that interacts with users on a regular basis. The robot was designed to handle repeated interaction while building a report with customers as well as providing useful information to them. The begins to answer questions about how people will respond to robots interacting in a commercial setting.
% /////////////////////////////////////////////////////////////////////////////////////////////
\section*{Critique}
% DISCUSS THE CRITIQUE OF THE PAPER ////////////////////////////////////////////////////////////
The main issue of the paper is the method in which data collection was done. Because they relied on RFID to remember people they needed participants to sign up. This causes problems because the people who sign up for a study are more likely to interact with the robot than others who do not sign up. This will skew the results of the study towards people who are interested in robotic help.
% /////////////////////////////////////////////////////////////////////////////////////////////
\cite{6256015}

\end{document}
