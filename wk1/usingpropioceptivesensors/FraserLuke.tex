\documentclass{article}
\usepackage{hyperref}
\usepackage{natbib}
\usepackage[right=1in,top=1in,left=1in,bottom=1in]{geometry}

\begin{document}
\title{{\large Review} \\ Using Proprioceptive Sensors for Categorizing Human-Robot Interactions}
\author{Luke Fraser}
\date{\today}
\maketitle

% REFERENCE THE PAPER HERE ////////////////////////////////////////////////////////////////////
\begingroup
\renewcommand{\section}[2]{}
\bibliographystyle{plain}
\bibliography{references}
\endgroup

% /////////////////////////////////////////////////////////////////////////////////////////////
\section*{Summary}
% WRITE  SUMMARY SECTION HERE /////////////////////////////////////////////////////////////////
In this paper the author's conduct an experiment and evaluate a heuristic for adaptive robotic behavior with children using their Roball toy. The Roball is a small toy ball that contains several sensors and motors that allow the ball to sense it's environment as well as maneuver itself. The first experiment that was conducted evaluated whether or not the information gathered from the sensors would provide enough information to understand what was happening to the Roball. After the authors were able to show that the Roball could be used to detect different events that happen during play they then apply different heuristics based on this data to adapt to different play events received from children. They classified the different interaction into 4 modes, Alone, general interaction, caring, and spinning. A fifth mode was also used to classify events outside the 4 other modes. The robot analyzed 4 second windows to determines it's current state and evaluated every .1 seconds. It was found that the accuracy of identifying the different modes varied. Detecting alone with 97\% accurate and the worst detecting general interactions 10\% accurate. The robot then responded differently depending on it's current state.
% /////////////////////////////////////////////////////////////////////////////////////////////
\section*{Strengths}
% DISCUSS THE STRENGTHS OF THE PAPER //////////////////////////////////////////////////////////
The experimental data was clearly shown and explained by the authors as well as identifying the several interactions they wanted to classify. Incorporating different responses to states of play could produce far for interactive toys/robots that provide better longer lasting interactions with kids.
% /////////////////////////////////////////////////////////////////////////////////////////////
\section*{Critique}
% DISCUSS THE CRITIQUE OF THE PAPER ////////////////////////////////////////////////////////////
The authors could have used more generalized methods to classify the different actions the robot went through. I believe they could have achieved better results if they had used learning algorithms to probabilistically classify the different modes the robot would experience. This would provide more accurate classifiers of the 4 modes as well as providing a general way to add more modes to be detected in the future.
% /////////////////////////////////////////////////////////////////////////////////////////////
\cite{6251676}

\end{document}
